\documentclass[mitschriften.tex]{subfiles}
\begin{document}
\subsection{}
Behauptung:
\begin{equation*}
  \mathbb{Z}_m^{l\times l} \cong  \mathbb{Z}_p^{l\times l} \times  \mathbb{Z}_q^{l\times l}
\end{equation*}
falls $m=p \cdot q$ mit ggt$(p,q) = 1$. Der zugehörige Isomorphismus $\pi$ sei
\begin{equation*}
  \pi:A \Rightarrow (A\mod p, A\mod q)\\
\end{equation*}
Weiter sei
\begin{align*}
  C &= A\cdot B\\
  c_{ij} &= \left(\sum_{k=1}{l}a_{ik}\cdot b_{kj}\right)\mod m 
\end{align*}
Also ist $c_{ij} \in \mathbb{Z}_p$ und $c_{ij} \in \mathbb{Z}_q$, also 1 Lösung für CRS.
\begin{equation*}
  \pi(A) = (A\mod p, A\mod q) \Rightarrow f(A^{-1}) = (A^{-1}\mod p, A^{-1}\mod q)
\end{equation*}
Für $m=6$ also (mit der Formel aus Aufgabe 17):
\begin{equation*}
  \|\mathbb{Z}_6^{l\times l} \| = \underbrace{\|\mathbb{Z}_2^{l\times l}\|}_{(2^2-1)(2^2-2)} \cdot  \underbrace{\|\mathbb{Z}_3^{l\times l}}_{(3^2-1)(3^2-3)}\| = 288
\end{equation*}
Analog für $m=26$:
\begin{equation*}
  \|\mathbb{Z}_{26}^{l\times l} \| = 157248
\end{equation*}
Für $m=8$:
\begin{align*}
  8^{-1} \mod 2 &\equiv 6\\
\end{align*}
\begin{equation*}
\exists A^{-1}\mod 8 \Leftrightarrow \text{ggT}(8, \det(A)) = 1 \Leftrightarrow \text{ggT}(2, \det(A)) = 1 \Leftrightarrow \exists A^{-1}\mod 2
\end{equation*}
\begin{equation*}
  \|\mathbb{Z}_8^{l\times l} \| = 4^4 
\end{equation*}
\subsection{}

\subsection{}
\subsection{}
\subsubsection{}
\begin{align*}
  x &= \underbrace{\framebox{\ldots}}_{l}\underbrace{\framebox{\ldots}}_{l}\underbrace{\framebox{\ldots}}_{l}\underbrace{\framebox{\ldots}}_{l}\\
  y &= \framebox{\ldots}\framebox{\ldots}\framebox{\ldots}\framebox{\ldots}\\
  K &= ? \in \mathbb{Z}_{p}^{l\times l}\\
  X &= ? \in \mathbb{Z}_{p}^{l\times l} \text{ falls  } X^{-1} \text{ existiert  }\\
  Y &= ? \in \mathbb{Z}_{p}^{l\times l} \text{ falls  } X^{-1} \text{ existiert  }\\
  X \cdot K &= Y\\
  \Rightarrow K &= X^{-1}Y\\
\end{align*}
Also ist $K$ einziger passender Schlüssel für die Blöcke.
% \begin{align*}
%   \begin{pmatrix}[ccc|ccc]
%     13 & 2 & 2 & 1 & 0 & 0\\
%     2 & 13 & 2 & 0 & 1 & 0\\
%     13 & 2 & 13 & 0 & 0 & 1\\
%   \end{pmatrix}
    %   \end{align*}
\subsection{}
\end{document}